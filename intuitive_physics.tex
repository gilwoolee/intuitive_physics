\documentclass{article}

% if you need to pass options to natbib, use, e.g.:
% \PassOptionsToPackage{numbers, compress}{natbib}
% before loading nips_2017
%
% to avoid loading the natbib package, add option nonatbib:

\usepackage[nonatbib, final]{nips_2017}

% to compㅁle a camera-ready version, add the [final] option, e.g.:
\usepackage[utf8]{inputenc} % allow utf-8 input
\usepackage[T1]{fontenc}    % use 8-bit T1 fonts
\usepackage{hyperref}       % hyperlinks
\usepackage{url}            % simple URL typesetting
\usepackage{booktabs}       % professional-quality tables
\usepackage{amsfonts}       % blackboard math symbols
\usepackage{nicefrac}       % compact symbols for 1/2, etc.
\usepackage{microtype}      % microtypography
\usepackage{amsmath, amssymb, amsfonts, amsthm}
\DeclareMathOperator*{\argmax}{arg\,max}
\DeclareMathOperator*{\argmin}{arg\,min}
\bibliographystyle{ieeetr}

\title{Bayesian-Adaptive Deep Reinforcement Learning via Ensemble Learning}

% The \author macro works with any number of authors. There are two
% commands used to separate the names and addresses of multiple
% authors: \And and \AND.
%
% Using \And between authors leaves it to LaTeX to determine where to
% break the lines. Using \AND forces a line break at that point. So,
% if LaTeX puts 3 of 4 authors names on the first line, and the last
% on the second line, try using \AND instead of \And before the third
% author name.

\author{
  Gilwoo Lee \\ \texttt{gilwoo@cs.uw.edu} \\
  %% examples of more authors
  \And
  Jeongseok Lee \\ \texttt{jslee02@cs.uw.edu} \\
  \And
  Brian Hou \\ \texttt{bhou@cs.uw.edu} \\
  \And
  Adtiya Vamsikrishna \\ \texttt{adityavk@cs.uw.edu} \\
}

\begin{document}
% \nipsfinalcopy is no longer used

\maketitle

\section{Introduction}
While reinforcement learning is capable of controlling complex autonomous systems, RL algorithms typically require huge amounts of data, can overfit to a particular task, or may learn brittle policies that are prone to disturbances. One of the main challenges that needs to be addressed is to train a policy that is robust to various model uncertainties and disturbances. In this project, we aim to address this challenge via an ensemble policy for Bayes-Adaptive Reinforcement Learning~\cite{ghavamzadeh2015bayesian}.

We assume that there exists a latent physics variable $\phi$ which determines the transition function of the underlying MDP, i.e. the transition function  $P(s',\phi' |s, \phi, a)$ is now a function of state, action, and $\phi$. We would like to learn a policy which maximizes the long term reward given $\phi$. Formally, this is called a Bayes-Adaptive MDP~\cite{ghavamzadeh2015bayesian, guez2012efficient}, defined by a tuple $<\mathcal{S}', \mathcal{A}, P, P_0, R>$ where
\begin{itemize}
\item $\mathcal{S'} = \mathcal{S}\times \Phi$ is the set of hyper-states (states, physics variable),
\item $\mathcal{A}$ is the set of actions,
\item $P(s',\phi'|s, \phi, a)$ is the transition function between hyper-states, conditioned
on action $a$ being taken in hyper-state $(s, \phi)$,
\item $P_0\in \mathcal{P}(\mathcal{S} \times \Phi)$ combines the initial distribution over hyper-states,
\item $R(s, \phi, a)$ represents the reward obtained when action $a$ is
taken in hyper-state $(s,\phi)$.
\end{itemize}

We would like to find the optimal policy for the following Bellman equaton:
\begin{equation}\label{eq:rl}
V^*(s, \phi) = \max_a \mathbb{E} \bigg[R(s, a, \phi) + \gamma \sum_{s', \phi'}P(s',\phi'|s, \phi, a)V^*(s', \phi') \bigg]
\end{equation}
This formulation is often referred to as Bayes-Adaptive Reinforcement Learning (BARL)~\cite{ghavamzadeh2015bayesian}.

We make a simplification to the BARL formulation. We assume that the dynamics of $s'$ and $\phi'$ are independent given $P(s, \phi, a)$, i.e.
\begin{equation*}
P(s',\phi'|s, \phi, a) = P(s'|s, \phi, a)\cdot P(\phi'|s, \phi, a).
\end{equation*}
This assumption allows us to simplify BARL with a gated ensemble policy learning method. At a high level, we have a gating network that determines the \emph{belief} of the physics parameters at time $t$,
\begin{align*}
b(\phi_t) = P(\phi_t|s_{t-1}, \phi_{t-1}, a_{t-1})
\end{align*}
which is then used to compute the best policy from an ensemble of $\phi$-dependent optimal policies, i.e., $\pi^*(\cdot;\phi)$ and $V^*(\cdot;\phi)$ are computed with typical RL algorithms for MDPs. Then the remaining task is to compute the one-step best action $a$:
\begin{align}\label{eq:barl}
 a^* &= \argmax_{a} \mathbb{E}_{\phi \sim b(\phi)} \bigg[R(s, a, \phi) + \gamma \sum_{s', \phi'}P(s',\phi'|s, \phi, a)V^*(s', \phi') \bigg]
\end{align}

We model $b_\phi$ as a network capable of modeling evolving state change, e.g. Recurrent Neural Networks, or as a Bayes filter. At the low level, we plan to discretize $\Phi$ and have one actor-critic network per discretized value of $\phi$: each critic estimates $V^*(\cdot;\phi)$ and each actor has an optimal policy for a particular discretized value of $\pi^*(\cdot;\phi)$. Given $b_\phi$ and the set of actor-critic networks, it is straightforward to compute (\ref{eq:barl}).

\section{Background}
Our work is closely related to QMDP~\cite{littman1995learning, karkus2017qmdp} which is an approximation for POMDP. QMDP approximates POMDP by assuming fully-observable MDP after 1-step, and approximating the Q-value at the current belief state $b(s)$ as $Q_a(b) =\sum_s b(s)Q_{MDP}(s, a)$. In our problem setup, we have a belief over the physics parameters $\phi$ of the MDP, $b(\phi)$, and we compute the policy $Q_a(s;b) = \sum_\phi b(\phi)Q_{MDP}(s,a;\phi)$.

The BAMDP formulation is also similar to POMDP formulation used in POMDP-lite~\cite{chen2016pomdp} which assumes that the hidden state variables are constant or only change deterministically. In our case, the hidden state variables correspond to the physics parameters $\phi$. The authors of POMDP-lite have shown that such formulation is ``equivalent to a set of fully observable Markov decision processes indexed by a hidden parameter''~\cite{chen2016pomdp}, which, in our case, could be viewed as a discretization of $\phi$.

\section{Platform}
First, we plan to validate our approach in simulation with a set of canonical examples such as the inverted pendulum, reacher, and hopper. We also plan to do physical robot experiments on the RACECAR platform to learn policies that are robust to different road conditions and poorly-estimated friction parameters between the car's wheels and the ground.

\section{Milestones}
\begin{itemize}
\item April 30 -- Set up all simulation environments in DART (including a kinematic simulation for the RACECAR)
\item May 15   -- Test algorithm on inverted pendulum, hopper, etc. in simulation
\item May 22   -- Test algorithm on RACECAR in simulation
\item May 31   -- Run experiments on the real RACECAR
\end{itemize}

\bibliography{intuitive_physics}

\end{document}
